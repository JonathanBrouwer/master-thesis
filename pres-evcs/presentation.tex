% !TeX document-id = {2870843d-1baa-4f6a-bd0a-a5c796104a32}
% !TeX encoding = UTF-8
% TU Delft beamer template

\documentclass[aspectratio=43]{beamer}
\usepackage{csquotes}
\usepackage{calc}
\usepackage[absolute,overlay]{textpos}
\usepackage{graphicx}
\usepackage{subfig}
\usepackage{mathtools}
\usepackage{amsfonts}
\usepackage{amsthm}
\usepackage{comment}
\usepackage{siunitx}
\usepackage{MnSymbol,wasysym}
\usepackage{array}
\usepackage{qrcode}

\setbeamertemplate{navigation symbols}{} % remove navigation symbols
\mode<presentation>{\usetheme[verticalbar=false]{tud}}

\newcommand{\absimage}[4][0.5,0.5]{%
	\begin{textblock}{#3}%width
		[#1]% alignment anchor within image (centered by default)
		(#2)% position on the page (origin is top left)
		\includegraphics[width=#3\paperwidth]{#4}%
\end{textblock}}

\newcommand{\mininomen}[2][1]{{\let\thefootnote\relax%
	\footnotetext{\begin{tabular}{*{#1}{@{\!}>{\centering\arraybackslash}p{1em}@{\;}p{\textwidth/#1-2em}}}%
	#2\end{tabular}}}}

\title[]{Dependently Typed Languages in Statix}
\institute[]{Delft University of Technology, The Netherlands}
\author{Jonathan Brouwer \and Jesper Cockx \and Aron Zwaan}

\begin{document}
\section{Introduction}
{
\setbeamertemplate{footline}{\usebeamertemplate*{minimal footline}}
\frame{\titlepage}
}

\begin{frame}[fragile]{Background: What are Dependent Types?}
\begin{itemize}
	\item Types may depend on values!
	\begin{exampleblock}{Example}
		\texttt{concat : (A: Set) -> (n : Nat) -> Vec A n -> Vec A n
			\\ \hspace*{48pt} -> Vec A (n + n)}
	\end{exampleblock}
% - n is a RUNTIME value, it does not have to be known at compile time
% - n occurs insie a type `Vec`!
% - The type system enforces these lengths are always correct
% - Challenge: Type checking may require evaluating arbitrary terms to decide equality
	\item Curry-Howard correspondence
% Can be used to write proofs about code
\end{itemize}


\end{frame}

\begin{frame}[fragile]{Research Question}
How well Statix is fit for the task of defining a dependently-typed language.
% Statix is turing complete, so it's going to be possible
% But is it going to be easier or harder than a general purpose language like haskell?
\end{frame}

\begin{frame}[fragile]{Why is this important?}
	\begin{block}{From the perspective of Spoofax research}
		Developing a language with a complex type system tests the boundaries of what Spoofax can do.
	\end{block}
	% - What can we improve about Statix?
	
	\begin{block}{From the perspective of Dependent Types research}
		A rapid prototyping platform.
	\end{block}
	% - Try a new feature without having to implement in general purpose
	% - Spoofax provides an easy way to get a parser, type checker, etc
\end{frame}

\begin{frame}[fragile]{Calculus of Constructions}
	A lambda calculus with dependent types.
	
	\begin{exampleblock}{Example 1}
		\texttt{($\backslash$v: Type. v) T}
	\end{exampleblock}

	\begin{exampleblock}{Example 2}
		\texttt{let f = $\backslash$T: Type. $\backslash$x: T. x; \\
f (T: Type -> Type) ($\backslash$y: Type. y)
		}
	\end{exampleblock}
\end{frame}

\begin{frame}[fragile]{Type Checking}
	\begin{block}{Type checking relation}
		\texttt{typeOfExpr : scope * Expr -> Expr}
	\end{block}
	% Returns an Expr, not a Type
	% Scope -> Two different types of nodes
	\begin{block}{How do we use scopes?}
		content...
	\end{block}
\end{frame}

\begin{frame}[fragile]{Extra contributions}
\begin{enumerate}
	\item Implemented Inference
	\item Implemented Inductive Data Types
	\item Implemented Universes
	\item Interpreter
	\item Compiler to Clojure
	\item Comparison with implementation in Haskell
	\item Comparison with implementation in LambdaPi
	\item Evaluation of Spoofax
\end{enumerate}
\end{frame}

% block
% exampleblock
% alertblock

\end{document}

