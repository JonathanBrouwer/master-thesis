\chapter{\label{chap:bools}Extending the language}

In this chapter, we will add booleans, postulates and type asserts to the language. The goal of this is to show that this language is easy to extend, and to add some features that make testing the language easier.

\section{Booleans}

This section describes how to add Booleans to the language. We will add the type of booleans \verb|Bool|, \verb|true|, \verb|false| and \verb|if|. The \verb|if| expression is not dependent, it expects both branches to have the same type. An example of a program with booleans: 

\begin{lstlisting}
let and = \x: Bool. \y: Bool. if x then y else false end
\end{lstlisting}
The constructors for the language are:

\begin{lstlisting}
BoolType : Expr
BoolFalse : Expr
BoolTrue : Expr
BoolIf : Expr * Expr * Expr -> Expr
\end{lstlisting}
Then, the rules for beta reducing the language are in figure \ref{fig:bool-rules-beta}. There is one particularly interesting case, that is how to beta-reduce if statements. Converting this rule to Statix is not entirely trivial, since you need to choose which rule to apply based on what \verb|c| evaluates to. A new rule is needed, which has 3 cases. One for if \verb|c| evaluates to true, one for if \verb|c| evaluates to false, and finally a third case for other cases (such as when \verb|c| is a variable that does not have a substitution). These rules are stated below: (Remember that \verb|rebuild| is the rule introduced in chapter \ref{chap:baselang}, which takes a scoped expression and a list of arguments and converts it to an expression by adding \verb|FnDestruct|s)

\begin{lstlisting}
betaReduceHead((s, BoolIf(c, b1, b2)), t) = 
  betaReduceHeadIf(s, betaReduceHead((s, c), []), c, b1, b2, t).
betaReduceHeadIf(s, (_, BoolTrue()), _, b1, _, t) = 
  betaReduceHead((s, b1), t).
betaReduceHeadIf(s, (_, BoolFalse()), _, _, b2, t) = 
  betaReduceHead((s, b2), t).
betaReduceHeadIf(s, _, c, b1, b2, t) = 
  rebuild((s, BoolIf(c, b1, b2)), t).
\end{lstlisting}

\begin{figure}[ht]
\begin{mathpar}
	\inferrule{
	} {
		\bhr
		{ \scope{s}{\co{BoolTrue}()} }
		{ [] }
		{ \scope{s}{\co{BoolTrue}()}}
	}

	\inferrule{
	} {
		\bhr
		{ \scope{s}{\co{BoolFalse}()} }
		{ [] }
		{ \scope{s}{\co{BoolFalse}()}}
	}

	\inferrule{
	} {
		\bhr
		{ \scope{s}{\co{BoolType}()} }
		{ [] }
		{ \scope{s}{\co{BoolType}()}}
	}

	\inferrule{
		\bhr
		{ \scope{s}{c} }
		{ [] }
		{ \scope{s'}{\co{BoolTrue}()} }
		\\\bhr
		{ \scope{s}{b1} }
		{ t }
		{ \scope{s''}{b1'} }
	} {
		\bhr
		{ \scope{s}{\co{BoolIf}(c, b1, b2)} }
		{ t }
		{ \scope{s}{b1'}}
	}

	\inferrule{
		\bhr
		{ \scope{s}{c} }
		{ [] }
		{ \scope{s'}{\co{BoolFalse}()} }
		\\\bhr
		{ \scope{s}{b2} }
		{ t }
		{ \scope{s''}{b2'} }
	} {
		\bhr
		{ \scope{s}{\co{BoolIf}(c, b1, b2)} }
		{ t }
		{ \scope{s}{b2'}}
	}
\end{mathpar}
\caption{Rules for beta head reducing booleans}
\label{fig:bool-rules-beta}
\end{figure}

Next, the rules for type-checking booleans are in figure \ref{fig:bool-rules-typecheck}, which are relatively simple. The if expression checks that both branches have the same type, as it is a non-dependent if statement.

\begin{figure}[ht]
\begin{mathpar}
	\inferrule{
	} {
		\toe{s}{\co{BoolType}()}{\co{Type}()}
	}

	\inferrule{
	} {
		\toe{s}{\co{BoolTrue}()}{\co{BoolType}()}
	}

	\inferrule{
	} {
		\toe{s}{\co{BoolFalse}()}{\co{BoolType}()}
	}

	\inferrule{
		\toe{s}{c}{ct}
		\\\beq{ct}{\co{BoolType}()}
		\\\\
		\\\toe{s}{b1}{tb1}
		\\\toe{s}{b2}{tb2}
		\\\beq{tb1}{tb2}
	} {
		\toe{s}{\co{BoolIf}(c, b1, b2)}{tb1}
	}
\end{mathpar}
\caption{Rules for type checking booleans}
\label{fig:bool-rules-typecheck}
\end{figure}

\section{Postulate}

Next we will add \verb|Postulate| to the language. A postulate declares that there is a variable with a certain type, without specifying a value. Through the view of the Curry-Howard correspondence, this is equivalent to an axiom. At the current stage, this is useful for testing the language. For example, this is a test with a postulate:
\begin{lstlisting}
postulate T : Type;
if true then Bool else T end
\end{lstlisting}
The following rules are used to implement the feature, which translate cleanly to Statix:

\begin{figure}[ht]
	\begin{mathpar}
		\inferrule{
			\bhr
			{\scope{s}{b}}
			{a}
			{\scope{s'}{b'}}
		}{
			\bhr
			{\scope{s}{\co{Postulate}(n, t, b)}}
			{a}
			{\scope{s'}{b'}}
		}
	
		\inferrule{
			\toe{s}{t}{tt}
			\\\beq{tt}{\co{Type}()}
			\\\\\bred{s}{t}{t'}
			\\\toe{\co{sPutType}(s, n, t')}{b}{bt}
		} {
			\toe{s}{\co{Postulate}(n, t, b)}{bt}
		}
	\end{mathpar}
	\caption{Rules for postulates}
	\label{fig:postulate-rules}
\end{figure}

\section{Type Assert}

Finally, we will add \verb|TypeAssert| to the language. This is another feature that is useful for testing. It takes an expression, and it asserts that the expression has a certain type. For example, we can do the following:
\begin{lstlisting}
postulate T : Type;
true : if true then Bool else T end
\end{lstlisting}

Implementing this feature is also straight-forward, we add the following two rules, which translate cleanly to Statix:

\begin{figure}[ht]
	\begin{mathpar}
		\inferrule{
			\bhr
			{\scope{s}{b}}
			{a}
			{\scope{s'}{b'}}
		}{
			\bhr
			{\scope{s}{\co{TypeAssert}(b, t)}}
			{a}
			{\scope{s'}{b'}}
		}
		
		\inferrule{
			\toe{s}{t}{tt}
			\\\beq{tt}{\co{Type}()}
			\\\\\toe{s}{b}{bt}
			\\\beq
			{bt}
			{\scope{s}{t}}
		} {
			\toe{s}{\co{TypeAssert}(b, t)}{bt}
		}
	\end{mathpar}
	\caption{Rules for type assertions}
	\label{fig:typeassert-rules}
\end{figure}


\section{Extensibility of the approach}

Now that a few features have been implemented, we can discuss how easy the language is to extend. The conclusion so far is that it is doable in a clean way. Specifically, the approach for each of the features above was:

\begin{enumerate}
	\item Define the parsing rules for the new feature
	\item Create a new file, \verb|tp_bools.stx| and import this file in the main \verb|type_check.stx| file.
	\item In the new file, define a case for the \verb|betaReduceHead| rule for the constructors that were added. Also define cases for \verb|expectBetaEq| and \verb|betaReduce| if necessary. It is necessary iff the new constructors are not always eliminated by \verb|betaReduceHead|.
	\item In the new file, define a case for the \verb|typeOfExpr| rule for the constructors that were added.
\end{enumerate}

This allows each feature to be isolated to its own file. If we decide that we don't like a feature after all, we can remove it simply by unimporting the file. This problem is similar to the expression problem\cite{expression_problem}, which Statix solves very well.

In the following chapters, we're going to be extending the language further with these more interesting features using this approach:
\begin{itemize}
	\item Inference (chapter \ref{ch:inference})
	\item Inductive Datatypes (chapter \ref{ch:datatypes})
	\item Universes (chapter \ref{ch:universes})
\end{itemize}



