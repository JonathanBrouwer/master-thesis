\chapter{Inductive Data Types}
\label{ch:datatypes}

Another useful feature is support for inductive data types. An example of a simple inductive datatype is the \verb|Nat| type:
\begin{lstlisting}
data Nat : -> Type where
    zero : Nat,
    suc : Nat -> Nat;
\end{lstlisting}
We will design data types to have the same features as in Agda:
\begin{enumerate}
\item Parameters, which are datatypes that are polymorphic, and are required to be the same for all constructors, such as
\begin{lstlisting}
data Maybe (T : Type) : -> Type where
    None : Maybe T,
    Some : T -> Maybe T;
\end{lstlisting}
\item Indices, which are datatypes that are polymorphic, that may vary from constructor to constructor
\begin{lstlisting}
data Eq : (e1: Bool) (e2 : Bool) -> Type where
    refl : e : Bool -> Eq e e;
\end{lstlisting}
\item We can also combine parameters and indices, for example to create the generic \verb|Eq| type. Note that parameters and indices may also depend on earlier parameters and indices.
\begin{lstlisting}
data Eq (T : Type) : (e1 : T) (e2 : T) -> Type where
    refl : e : T -> Eq e e;
\end{lstlisting}
\item Data types must be strictly positive. This ensures that we cannot make non-terminating programs\footnote{After we add support for universes in chapter \ref{ch:universes}}. For example, the following data type is forbidden, because it refers to itself in a negative position. We will explain exactly how this check works in section \ref{ch:datatypes:positivity}.
\begin{lstlisting}
data Bad : -> Type where
	bad : (Bad -> Bool) -> Bad;
\end{lstlisting}

\section{In statix}



\section{Positivity Checking}
\label{ch:datatypes:positivity}
	
\end{enumerate}