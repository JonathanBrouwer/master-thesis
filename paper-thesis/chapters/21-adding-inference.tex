\chapter{Term Inference}
\label{ch:inference}

Inference is an important feature of dependent programming languages, that allows redundant parts of programs to be left out. For example, it allows you to infer the arguments of a function, if they can be inferred by the arguments that follow, like here where the type of the argument can be inferred, since we pass it \verb|true| which is a boolean:
\begin{lstlisting}
let id = (\T : Type. \x: T. x);
id _ true
\end{lstlisting}

We call it \emph{term inference} rather than \emph{type inference}, because we can infer values other than types. For example, it can infer that the \verb|_| in this example must be \verb|true|:
\begin{lstlisting}
postulate f: Bool -> Type;
postulate g: f true -> Type;
\x: f _. g x
\end{lstlisting}

\section{Different algorithms for inference}
\label{strength-inference}

There are a lot of different algorithms for inference\cite{typeinference}, some algorithms can solve more inferences than others. One algorithm for unification is \emph{first-order unification}, where if at any point during type checking we assert that $\beq{e_1}{e_2}$ and either $e_1$ or $e_2$ is a free variable, we set the the value free variable to be equal to the value of the non-free variable. There are some situations in which this approach fails, but in most real-world scenarios it works perfectly. For example, it can infer both programs in the introduction of this chapter, but it fails to infer the following program:

\begin{lstlisting}
let f = _;
\x : Type.
\g: (_: (f x) -> Bool).
g true
\end{lstlisting}

We know that \verb|f| is a function from \verb|Type -> Type|, but it fails to infer the value of \verb|f|. Because of the way that \verb|g| is used, the type checker asserts that $\beq{f x}{Bool}$. Since x is declared as a function argument and it is completely free, this means that for any \verb|x|, \verb|f x = Bool|. But the rule above is not powerful enough to derive this, so it fails.

\section{Inference in Statix}
\label{statix-inference}

We would like to avoid implementing an algorithm at all, instead using Statix' built-in first-order unification to do the type inference for us. Implementing an inference algorithm in Statix is theoretically possible but this would be a lot of ugly code (since Statix is not a general-purpose programming language), and the goal is to use Statix in a way that is clean and declarative, not to do optimal inference.

However, we cannot immediately use Statix' built-in first-order unification (which acts in the meta language, Statix) to implement first-order unification in the object language. Ideally when implementing beta equality we would match on $\beq{e_1}{e_2}$ where $e_1$ is a free variable, but Statix does not allow for querying whether variables are free. 

Instead, we will be implementing a novel, less powerful form of first-order unification. This will work by explicitly denoting which variables \emph{could be} free, and explicitly handling these cases in a way that approximates first-order unification. We will denote this algorithm as \emph{approximated first-order unification (AFOU)}.

\section{Implementing AFOU}
\label{implementing-inference}

First, we introduce a new constructor \verb|Infer : Expr -> Expr|, which denotes the variables which could be free. The constructor is introduced when we encounter a \verb|_| variable, a marker that something needs to be inferred.

\begin{lstlisting}
typeOfExpr(s, Var(Syn("_"))) = (Infer(q), qt) :-
	(_, qt) == typeOfExpr(sEmpty(), q).
\end{lstlisting}

Note that the type of \verb|typeOfExpr| has changed, it is now 

\begin{lstlisting}
typeOfExpr : scope * Expr -> Expr * Expr
\end{lstlisting}

\section{Analysis of the power of AFOU}
\label{analysing-inference}


\begin{comment}
	

\section{Using Statix' first-order unification for inference}



When an \verb|Infer| needs to be type-checked, the logic is very similar, except that we don't need to generate a new metavariable:

\begin{lstlisting}
typeOfExpr_(s, Infer(q)) = (Infer(q), t) :-
	typeOfExpr_(s, q) == (_, t).
\end{lstlisting}
When we encounter a \verb|Infer| in \verb|betaReduceHead|, we keep it intact, because we still want to know that it is a \verb|Infer| in the \verb|expectBetaEq| rule. 

The \verb|expectBetaEq| is where it gets interesting. We have a rule for each constructor in the language. For simple cases like \verb|BoolTrue|, we assert that \verb|e2| must be equal.
\begin{lstlisting}
expectBetaEq_((s1, e1@BoolTrue()), (_, Infer(e2))) :- e1 == e2.
\end{lstlisting}
For more complicated constructors such as \verb|FnType|, we generate new metavariables for the subexpressions, and assert they it must be beta-equal to the provided ones:
\begin{lstlisting}
expectBetaEq_(
	(s1, e1@FnType(arg_name1, arg_type1, body1)), 
	(_, Infer(e2))) :- {arg_type2 body2}
  e2 == FnType(arg_name1, Infer(arg_type2), Infer(body2)),
  expectBetaEq_((s1, e1), (sEmpty(), e2)).
\end{lstlisting}

Finally, there is a case for if we encounter two \verb|Infer|s. Ideally, we would look at whether one infer is instantiated, and apply the rules above. However, this is not possible since we cannot do such queries in Statix. Instead, we say that they must be exactly equal and we hope for the best. There are situations where this rule fails (where e1 and e2 are beta-equal but not identical), but this works well enough for practical use.
\begin{lstlisting}
expectBetaEq_((_, Infer(e1)), (_, Infer(e2))) :-
    e1 == e2.
\end{lstlisting}
\end{comment}