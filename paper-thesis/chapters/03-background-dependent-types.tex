% !TEX root = document.tex
\chapter{\label{chap:bg-dp}Background: Dependent Types}

This chapter will explain the concepts behind dependent types. Readers already familiar with Dependent Types are recommended to only skim through this chapter.

\section{What are dependent types?}

A dependent type system is a type system in which types may depend on values. This means it is possible to have a type such as \verb|Vec n|, which is a vector of exactly length \verb|n|. Note that the value of \verb|n| does not have to be known at compile time, it is more powerful than that. 

The biggest advantage of dependent types is that they increase the expressiveness of a type system a lot, for example it is now possible to express a function that adds two vectors of the same length and returns a new vector of that length. 
\begin{lstlisting}
add_vec : (n : Nat) -> Vec n -> Vec n -> Vec n
\end{lstlisting}

\section{The Curry-Howard Isomorphism}

The Curry-Howard correspondence is a relationship that can be used to interpret typed programs as mathematical proofs~\cite{chc}. This is done by representing false statements as empty types, and true statements as non-empty (inhabited) types. 

For example, it is possible to define a type \verb|IsEven n|, that represent the statement "\verb|n| is even", and declare it in such a way that it is only inhabited if \verb|n| is even.

In order to use a type system to prove things it is important that the type system is sound, that is, it is impossible to prove a false statement. Using the Curry-Howard correspondence, that means it should be impossible to obtain an element of the empty type. We will explore this further in chapter \ref{ch:universes}.

\section{The Lambda Cube}

The lambda cube is a framework to categorize programming languages based on whether terms and types can depend on each other~\cite{lambda_cube}. Specifically, it categorizes languages on three axis: \footnote{Note that the "Terms can depend on terms" axis is missing, this is because a language where terms cannot depend on terms cannot compute anything and is thus pretty useless.}

\begin{enumerate}
	\item Terms can depend on types. This corresponds to type-polymorphic (generic) functions. For example, it allows us to define the polymorphic identity function:
	\begin{lstlisting}
id : (T : Type) -> T -> T
	\end{lstlisting}
	\item Types can depend on types. This corresponds to type-polymorphic (generic) datatypes. For example, it allows to the define the \verb|List T| type, a list with items of type T.
	
	\item Types can depend on terms. This corresponds to dependent types.
\end{enumerate}

The \textit{Calculus of Constructions} is the language that has all three of these features, and it is what we will implement in this paper. If you want to learn more, I recommend playing around with the \textit{\href{https://agda.readthedocs.io/en/latest/}{Agda}} language, which is a real-world language that satisfies all three axis.
