\chapter{Universes}
\label{ch:universes}

Our language so far has the property that the type of \verb|Type| is \verb|Type|, this is called \emph{type in type}. This allows for \emph{Girard's paradox}, which means that the logic created by this language is inconsistent: we can prove false. 

\section{Universes}

The solution to this problem is introducing universes into the language. We now have the following hierarchy of types:
\begin{lstlisting}
true : Bool : Type 0 : Type 1 : Type 2 : ...
\end{lstlisting}

In the rest of this chapter we will show how universes have been implemented. It turns out to be relatively easy to add this feature to the language, only requiring a few changes.

\section{Implementing universes in the CoC}

The constructor of Type has been changed to \verb|Type : int -> Expr|. We need to update all occurrences of \verb|Type| in figure \ref{fig:type-check-rules}. Figure \ref{fig:type-check-rules-universes} contains all the rules that require a change.

The three rules that were changed are:
\begin{enumerate}
	\item The rule for type-checking a \verb|Type| expression, which now is explicit about the universes of each \verb|Type|.
	\item The rule for type-checking \verb|FnType|, which takes the maximum of the two arguments.
	\item The rule for type-checking \verb|FnConstruct|, which ignores the universe of its argument type, and only verifies that it is a type.
\end{enumerate}

\begin{figure}[ht]
	\figuresection[\fbox{$\toe{s}{e}{t}$}]{Type checking rules with universes}
	\begin{mathpar}
		\mprset{vskip=0.7ex}
		% Type rule
		\inferrule{
		} {
			\toe{s}{\co{Type}(u)}{\co{Type}(u+1)}
		}
		
		% FnType rule
		\inferrule{
			\toe{s}{a}{t_a}
			\\ \beq{t_a}{\co{Type}(u_a)}
			\\ \bred{s}{a}{a'}
			\\\\ \toe{\co{sPutType}(s, x, a')}{b}{t_b}
			\\ \beq{t_b}{\co{Type}(u_b)}
		} {
			\toe{s}{\co{FnType}(x, a, b)}{\co{Type}(max(u_a, u_b))}
		}
		
		% FnConstruct rule
		\inferrule{
			\toe{s}{a}{t_a}
			\\ \beq{t_a}{\co{Type}(u_a)}
			\\ \bred{s}{a}{a'}
			\\\\ \toe{\co{sPutType}(s, x, a')}{b}{t_b}
		} {
			\toe{s}{\co{FnConstruct}(x, a, b)}{\co{FnType}(x, a', t_b)}
		}
		
		\mprset{vskip=0ex}
	\end{mathpar}
	\caption{Rules for type checking the Calculus of Constructions with universes}
	\label{fig:type-check-rules-universes}
\end{figure}

\newpage
\section{Implementing universes with inductive data types}

We change data type declarations such that the universe the datatype lives in is now explicit. For example, the \verb|Nat| type now looks like this, explicitly stating that the type of \verb|Nat| is \verb|Type 0|.
\begin{lstlisting}
data Nat : -> Type 0 where
	zero : Nat,
	suc : Nat -> Nat;
\end{lstlisting}

We need to make the following changes to the type-checking algorithm:
\begin{enumerate}
	\item In point \ref{name-decl} of type-checking data type declarations in section \ref{ch:datatypes:typechecking}, we need to return \verb|Type u| instead of \verb|Type|, where \verb|u| is the declared universe level.
	\item For all constructors, we need to check that the universe of the constructor arguments is smaller than or equal to \verb|u|. This ensures that the datatype does actually live in the universe \verb|u|.
\end{enumerate}

