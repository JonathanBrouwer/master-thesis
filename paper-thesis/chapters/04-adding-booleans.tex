\chapter{\label{chap:bools}Booleans and other "simple" constructs that really need a better title}

In this chapter, we will add booleans, postulates and type asserts to the language. The goal of this is to show that this language is easy to extend, and to add some features that make testing the language easier.

\section{Booleans}

This section describes how to add Booleans to the language. We will add the type of booleans \verb|Bool|, \verb|true|, \verb|false| and \verb|if|. The \verb|if| expression is not dependent, it expects both branches to have the same type. An example of a program with booleans: 

\begin{lstlisting}
let and = \x: Bool. \y: Bool. if x then y else false end
\end{lstlisting}
The constructors for the language are:

\begin{lstlisting}
BoolType : Expr
BoolFalse : Expr
BoolTrue : Expr
BoolIf : Expr * Expr * Expr -> Expr
\end{lstlisting}
Then, the rules for beta reducing the language are in figure \ref{fig:bool-rules-beta}. There is one particularly interesting case, that is how to beta-reduce if statements. Converting this rule to Statix is not entirely trivial, since you need to choose which rule to apply based on what \verb|c| evaluates to. A new rule is needed, which has 3 cases. One for if \verb|c| evaluates to true, one for if \verb|c| evaluates to false, and finally a third case for other cases (such as when \verb|c| is a variable that does not have a substitution). Remember that \verb|rebuild| is the rule introduced in chapter \ref{chap:baselang}, which takes a scoped expression and a list of arguments and converts it to an expression by adding \verb|FnDestruct|s. These rules are stated below:

\begin{lstlisting}
betaReduceHead((s, BoolIf(c, b1, b2)), t) = 
  betaReduceHeadIf(s, betaReduceHead((s, c), []), c, b1, b2, t).
betaReduceHeadIf(s, (_, BoolTrue()), _, b1, _, t) = 
  betaReduceHead((s, b1), t).
betaReduceHeadIf(s, (_, BoolFalse()), _, _, b2, t) = 
  betaReduceHead((s, b2), t).
betaReduceHeadIf(s, _, c, b1, b2, t) = 
  rebuild((s, BoolIf(c, b1, b2)), t).
\end{lstlisting}

\begin{figure}[ht]
	\begin{mathpar}
		\inferrule{
		} {
			\bhr
			{ \scope{s}{\co{BoolTrue}()} }
			{ [] }
			{ \scope{s}{\co{BoolTrue}()}}
		}
	
		\inferrule{
		} {
			\bhr
			{ \scope{s}{\co{BoolFalse}()} }
			{ [] }
			{ \scope{s}{\co{BoolFalse}()}}
		}
	
		\inferrule{
		} {
			\bhr
			{ \scope{s}{\co{BoolType}()} }
			{ [] }
			{ \scope{s}{\co{BoolType}()}}
		}
	
		\inferrule{
			\bhr
			{ \scope{s}{c} }
			{ [] }
			{ \scope{s'}{BoolTrue()} }
			\\\bhr
			{ \scope{s}{b1} }
			{ t }
			{ \scope{s''}{b1'} }
		} {
			\bhr
			{ \scope{s}{\co{BoolIf}(c, b1, b2)} }
			{ t }
			{ \scope{s}{b1'}}
		}
	
		\inferrule{
			\bhr
			{ \scope{s}{c} }
			{ [] }
			{ \scope{s'}{BoolFalse()} }
			\\\bhr
			{ \scope{s}{b2} }
			{ t }
			{ \scope{s''}{b2'} }
		} {
			\bhr
			{ \scope{s}{\co{BoolIf}(c, b1, b2)} }
			{ t }
			{ \scope{s}{b2'}}
		}
	\end{mathpar}
	\caption{Rules for beta head reducing booleans}
	\label{fig:bool-rules-beta}
\end{figure}


\section{Postulate}

Next we will add \verb|Postulate| to the language.

\section{Type Assert}

Finally, we will add \verb|TypeAssert| to the language.