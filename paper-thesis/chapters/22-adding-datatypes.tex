\chapter{Inductive Data Types}
\label{ch:datatypes}

\section{Introduction to Inductive Data Types}
\label{ch:datatypes:intro}

Another useful feature is support for inductive data types. Inductive data types declare types that are defined by a list of constructors that produce the type. An example of a simple inductive datatype is the \verb|Nat| type:
\begin{lstlisting}
data Nat : -> Type where
    zero : Nat,
    suc : Nat -> Nat;
\end{lstlisting}
The data type has two constructors: \verb|zero| represents the natural number zero, \verb|suc| takes a natural number and represents the successor of that number. For example, \verb|suc (suc zero)| represents 2.


We will design data types to have the same features as in Agda:
\begin{enumerate}
\item Data types may be recursive, that is the data type may take itself as one of the constructor arguments. This can be seen in the definition of \verb|Nat| above.
\item Data types may have \emph{parameters}, which are datatypes that are polymorphic over a certain value, that is required to be the same for all constructors, such as:
\begin{lstlisting}
data Maybe (T : Type) : -> Type where
    None : Maybe T,
    Some : T -> Maybe T;
\end{lstlisting}
\item Data types may have \emph{indices}, which are datatypes that are polymorphic over a certain value, that may vary from constructor to constructor, such as:
\begin{lstlisting}
data Eq : (e1: Bool) (e2 : Bool) -> Type where
    refl : e : Bool -> Eq e e;
\end{lstlisting}
\item We can also combine parameters and indices, for example to create the polymorphic \verb|Eq| type, which represents a proof that two values are equal. Note that the type of parameters and indices may depend on the value earlier parameters and indices. For example, \verb|e1| and \verb|e2| have type \verb|T|.
\begin{lstlisting}
data Eq (T : Type) : (e1 : T) (e2 : T) -> Type where
    refl : e : T -> Eq e e;
\end{lstlisting}
\item Data types must be strictly positive. This ensures that we cannot make non-terminating programs\footnote{After we add support for universes in chapter~\ref{ch:universes}}. For example, the following data type is forbidden, because it refers to itself in a negative position. We will explain exactly how this check works in section \ref{ch:datatypes:positivity}.
\begin{lstlisting}
data Bad : -> Type where
	bad : (Bad -> Bool) -> Bad;
\end{lstlisting}
\end{enumerate}

A difference between Agda and our implementation is that Agda has case matching as a native construct, whereas we chose to use eliminators as described in Inductive families~\cite{eliminators}. An eliminator is a function that can be used to case match on a data type. For example, the eliminator of the \verb|Nat| type above is:
\begin{lstlisting}
elim Nat : P: (v: Nat -> Type) -> P Z 
	-> (x: Nat -> P x -> P (S x)) -> n: Nat -> P n
\end{lstlisting}

The general type of an eliminator for a data type \verb|N| is:
\begin{lstlisting}
parameters 
-> P: (indices -> v: N params indices -> Type) 
-> constructors
-> indices
-> v: N params indices
-> P indices v
\end{lstlisting}
The meaning of all the arguments is:
\begin{itemize}
	\item \verb|parameters| are the parameters of the data type we want to eliminate, since these are constant among all constructors those should be specified at the start.
	\item \verb|P| is the type that this eliminator will return. It may depend on the value \verb|v| that is eliminated.
	\item \verb|constructors| is a function that eliminates each of the constructors of the datatype. To eliminate a constructor of the form \verb|C : args -> N params indices| it generates a function \verb|args -> P indices (C pars args)|. For recursive datatypes, a previous case is generated, such as the \verb|P x| in the eliminator of \verb|Nat| above.
	\item \verb|indices| are the indices of the value that is to be eliminated.
	\item \verb|v| is the value that is to be eliminated.
	\item Finally, \verb|P indices v| is the result of the elimination.
\end{itemize}

\section{Type-checking Data Type Declarations}
\label{ch:datatypes:typechecking}
In this section we will explain how to type check data type declarations using Statix.
The definition of \verb|type_check| for an inductive data type \verb|N| is, conceptually:
\begin{enumerate}
	\item Create a new scope \verb|s1| whose parent is the scope the declaration was in \verb|s0|.
	\item \label{name-decl}Declare \verb|N : Params -> Indices -> Type| in \verb|s1|.
	\item \label{datatype-decl}We use a new scope-graph relation \verb|datatype : ID -> Expr| and declare \\ \verb|N : DataType(name, params, indices, constructors)| in \verb|s1| so that we can access information about the data type later.
	
	\item Create a chain of scopes starting in \verb|s1| that will contain a scope for each parameter. We will type-check each next parameter using the previous parameters scope, so that parameters can depend on previous parameters. Call the end of this chain \verb|s2|.
	
	\item Create a chain of scopes starting in \verb|s2| that will contain a scope for each index. We will type-check each next index using the previous index' scope, so that one index can depend on parameters and previous indices. Call the end of this chain \verb|s3|. The scope \verb|s3| is not used in the rest of this process, but creating this chain is required to check the types of the indices.
	
	\item Create a new scope \verb|s4| with \verb|s1| as the parent. Then type-check each constructor with \verb|s2| as scope (so that the constructors can depend on parameters, but cannot depend on indices), and declare the type of that constructor in \verb|s4|. The parameters are automatically added to the type of the constructor.
	
	\item Type-check the rest of the program with scope \verb|s4|.
\end{enumerate}
For example, figure \ref{scope-graph-eq} shows the scopes that were created during the type checking of the following inductive datatype:
\begin{lstlisting}
data Eq (T : Type) : (e1 : T) (e2 : T) -> Type where
	refl : e : T -> Eq e e;
\end{lstlisting}
\begin{figure}[!ht]
	\begin{framed}
		\centering
		\begin{tikzpicture}[scopegraph, node distance = 6.5em and 5em]
			% S0 and S1 node
			\node[scope] (s0) {$s0$};
			\node[scope, below = 2em of s0] (s1) {$s1$};
			\draw (s1) edge[lbl={P}] (s0);

			% S1 declarations
			\node[type, above right=0.5em and 3em of s1, text width=11cm] (ntype) {\texttt{Eq : (T : Type) -> (e1 : T) -> (e2 : T) -> Type)}};
			\draw (ntype) edge[type, lbl={name}] (s1);
			\node[type, below right=0em and 3em of s1, text width=11cm] (ndatatype) {\texttt{Eq : DataType("Eq", [("T", Type())], [("e1", Var("T")), ("e2", Var("T"))], [("refl", ...)])}};
			\draw (ndatatype) edge[type, lbl={datatype}] (s1);
			
			% Path to S3
			\node[scope, below right=4em and 1em of s1] (s2a) {s2};
			\draw (s2a) edge[lbl={P}] (s1);
			\node[type, right=3em of s2a] (ntype) {\texttt{T : Type}};
			\draw (ntype) edge[type, lbl={name}] (s2a);
			
			\node[scope, below=1em of s2a] (s2b) {};
			\draw (s2b) edge[lbl={P}] (s2a);
			\node[type, right=3em of s2b] (ntype) {\texttt{e1 : T}};
			\draw (ntype) edge[type, lbl={name}] (s2b);
			
			\node[scope, below=1em of s2b] (s2c) {s3};
			\draw (s2c) edge[lbl={P}] (s2b);
			\node[type, right=3em of s2c] (ntype) {\texttt{e2 : T}};
			\draw (ntype) edge[type, lbl={name}] (s2c);
			
			% S4
			\node[scope, below = 12em of s1] (s4) {$s4$};
			\draw (s4) edge[lbl={P}] (s1);
			\node[type, right=3em of s4] (ntype) {\texttt{refl : (T : Type) -> (e : T) -> Eq e e}};
			\draw (ntype) edge[type, lbl={name}] (s4);
		\end{tikzpicture}
	\end{framed}
	\caption{The scope graph generated by the Eq data type.}
	\label{scope-graph-eq}
\end{figure}

\section{Type-checking Eliminators}

This section discusses how to type check eliminators. The type of an eliminator was already discussed in section \ref{ch:datatypes:intro}. When we type check an \verb|elim N| expression, the following happens:
\begin{enumerate}
	\item Query the scope to find the data type declaration that we created in point \ref{datatype-decl} of section \ref{ch:datatypes:typechecking}. This gives us access to the parameters, indices and constructors of the data type that is being eliminated. 
	\item Using some tedious but conceptually simple Statix rules, create the type discussed in section \ref{ch:datatypes:intro}. This ends up being around 100 lines of Statix code.
\end{enumerate}

The second part that needs to be defined is beta reducing an eliminator. During this process, the following happens:
\begin{enumerate}
	\item Query the scope to find the data type declaration that we created in point \ref{datatype-decl} of section \ref{ch:datatypes:typechecking}. This gives us access to the parameters, indices and constructors of the data type that is being eliminated. (Same as during type checking)
	\item Try to split the arguments applied to the eliminator into the groups defined in section \ref{ch:datatypes:intro}. If this fails (because there are not enough arguments), the eliminator cannot be reduced.
	\item Beta head reduce \verb|v|, if this does not result in one of the constructors of the datatype at the head then the eliminator cannot be reduced. 
	\item Apply the relevant function that eliminates the constructor to the arguments of the datatype. Generate recursive calls to the eliminator for recursive datatypes.
\end{enumerate}

This is all that needs to happen to define inductive data types. Next, we will add positivity checking.

\section{Positivity Checking}
\label{ch:datatypes:positivity}
	
When defining a recursive data type, we would like our datatype to be \emph{strictly positive}, meaning that it can only be recursive on positive positions. If this rule would not exist, we would be able to create non-terminating functions using the datatype. Since under the curry-howard correspondence this allows to prove false, this is undesirable.

To show how this is can go wrong, consider the following example:
\begin{lstlisting}
data False : Set where

data Bad : Set where
	bad : (Bad → False) → Bad;

self-app : Bad → False
self-app (bad f) = f (bad f)

absurd : False
absurd = self-app (bad self-app)
\end{lstlisting}

Under the curry-howard correspondence, \verb|Bad| states that \verb|Bad| is true if and only if \verb|Bad| implies false. Since such a statement is inconsistent, we can use it to prove \verb|False|. To prevent such statements, we don't allow inductive datatypes to refer to themselves negatively, checking this is called \emph{positivity checking}. This solution is not complete (it rejects valid data types), but it is the solution most often used in practice, such as in Agda~\cite{agda}, Lean~\cite{lean} and Coq~\cite{coq}.

The exact condition that is checked is that if a constructor has a function as its argument, then the argument type (\verb|A| in the example below) can not refer to the declared data type.

\begin{lstlisting}
data Bad : Set where
	bad : (Bad -> Bad) -> Bad
	--     A       B       C
\end{lstlisting}

The implementation is simply an extra check during the type checking of data type declarations, which is straight-forward to implement in Statix.

We have now shown how to add inductive datatypes to the implementation. Next, we will add  Universes (chapter~\ref{ch:universes}) to the implementation.
