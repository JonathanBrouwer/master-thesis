\chapter{Conclusion}

This thesis presents an implementation of a dependently typed language in Statix. Our aim was to anther the following research questions:

\paragraph{RQ1: Can the Calculus of Constructions be implemented in Statix?}
As we demonstrated in chapter \ref{chap:baselang}, the Calculus of Constructions can be implemented concisely in Statix, by storing substitutions in the scope graph. We defined the beta-reduction, beta-equality and type checking rules and then converted them to Statix code. Beta reduction was defined using a Krivine machine. In chapter \ref{chap:namecolls} we solved the variable capture problem which the naive implementation from chapter \ref{chap:baselang} suffered from, by using scoped names.

\paragraph{RQ2: Is the implementation is easily extendable?}
We showed in chapter \ref{chap:bools} that the implementation is easily extendable, by extending it with booleans, postulate, and type assertions. We define a four step process that can be used to extend the language, which we also use to answer the remaining research questions.

\paragraph{RQ3: Can we add inference to the implementation?}
In chapter \ref{ch:inference} we discuss how to add inference to the implementation. If we want to keep the implementation clean and concise, we need to compromise on how powerful the inference algorithm is, by defining an approximated version of first-order inference. We implemented this algorithm in a concise way.

\paragraph{RQ4: Can we add support for inductive data types to the implementation?}
Chapter \ref{ch:datatypes} adds support for inductive data types with parameters and indices. We give steps for type-checking data type declarations. Additionally, we give each data type an eliminator and show how to type-check and beta reduce eliminators. Finally, we show that positivity checking can be implemented concisely.


\paragraph{RQ5: Can we add support for universes to the implementation?}
In chapter \ref{ch:universes} we showed that we can add universes to the language. This can be done easily and concisely.

\section{Future Work}
While the language as implemented currently is fully usable, there are still some open questions. 