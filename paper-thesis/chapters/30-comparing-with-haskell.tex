\chapter{A comparison with conventional implementations}
\label{ch:comp-haskell}

In this chapter, we implement the language defined in chapter~\ref{chap:baselang} in Haskell, and then compare the implementation with the implementation in Statix. We want the design of the implementation in Haskell to be representative of conventional implementations of dependently typed languages, so we can use it to compare the Statix implementation with them.

\section{Defining the AST}
To implement the calculus of constructions in Haskell, we first define the inductive datatype \verb|Expr|. We chose to define the language using De Bruijn indices, since this is the convention when implementing dependently typed languages, and the goal of this implementation in Haskell is to compare the Statix implementation with conventional ones. 
\begin{lstlisting}
data Expr =
	Type
	| Let Expr Expr
	| Var Int
	| FnType Expr Expr
	| FnConstruct Expr Expr
	| FnDestruct Expr Expr
\end{lstlisting}
There is a constructor for each constructor in our language. The difference with the implementation in Spoofax is that the \verb|Let|, \verb|FnType|, and \verb|FnConstruct| no longer store an identifier, instead they introduce a new binding into their body, which are accessible by their De Bruijn indices. 

\section{Defining Environments}

Next, we need to decide how we want to store the environment. We still need to store both function arguments and substitutions, which we called \verb|NType| and \verb|NSubst| in chapter~\ref{chap:baselang}. We will use the same names, and store the environment as a list, with the head of the list representing De Bruijn index 0. Finally, we define the type of a scoped expression \verb|SExpr|, as a tuple of an environment and an expression.
\begin{lstlisting}
data EnvEntry = NType Expr | NSubst SExpr
type Env = [EnvEntry]
type SExpr = (Env, Expr)
\end{lstlisting}

The way these environments are defined is isomorphic to the way we defined the way we use scope graphs in section \ref{sec:coc-scopes}. Nodes in the scope graph have at most one parent, and each node stores one entry, which is exactly the structure of a list. For example, the following scope graph and list have the same meaning:

\begin{lstlisting}
NType(Var(0)) :: NType(Type()) :: Nil
\end{lstlisting}
\begin{tikzpicture}[scopegraph, node distance = 2.5em and 2.5em]
	\node[scope] (s0) {$s0$};
	
	\node[scope, right of = s0, xshift=5em] (s1) {$s1$};
	\node[type, above of = s1] (s1l) {$\texttt{T : Type()}$};
	\draw (s1) edge[type] (s1l);
	
	\node[scope, right of = s1, xshift=5em] (s2) {$s2$};
	\node[type, above of = s2] (s2l) {$\texttt{x : Var("T")}$};
	\draw (s2) edge[type] (s2l);
	
	\draw (s1) edge[] (s0);
	\draw (s2) edge[] (s1);
\end{tikzpicture}

The nodes in scope graphs may have multiple children, but we never query the children of a node. We only follow the edges, we don't go in the opposite direction. Similarly, part of a list may be shared, but this is fine in Haskell, since values are immutable. 

Finally, we define the two functions \verb|sPutSubst| and \verb|sPutType| to mimic the Statix relations with the same name.
\begin{lstlisting}
sPutSubst :: Env -> SExpr -> Env
sPutSubst env v = NSubst v : env
sPutType :: Env -> Expr -> Env
sPutType env v = NType v : env
\end{lstlisting}

\section{Defining Type Checking}

Next we need to define type checking. We define a function \verb|tc|:
\begin{lstlisting}
tc :: SExpr -> Either String Expr
\end{lstlisting}

The function returns a \verb|String| in case of an error, or a \verb|Expr| which is the type of the expression. This is different than in Statix, where errors are automatically created during type checking by Statix. We use the power of the \verb|Either String| monad to our advantage, so we can define our function using \emph{do blocks}. For example, compare the implementation of \verb|Let| in figure \ref{fig:type-check-rules} (repeated below for convenience) with the implementation in Haskell:

\begin{lstlisting}
tc (s, Let e b) = do
	_ <- tc (s, e)
	bt <- tc (sPutSubst s (s, e), b)
	pure (shift (-1) 0 bt)
\end{lstlisting}

\begin{mathpar}
		\inferrule{
	\toe{s}{e}{t_e}
	\\ \toe{\co{sPutSubst}(s, x, (s, e))}{b}{t_b}
}{
	\toe{s}{\co{Let}(x, e, b)}{t_b}
}
\end{mathpar}

We are using do blocks to pass errors through the type checking function. The only significant difference with the inference rule is that we need to shift the resulting type by one, since the binding of the let-bound variable does not exist at the type level. This shifting is required for variables too, other than this the inference rules are converted directly to cases of the Haskell function.

\section{Defining Beta Reduction}

During type checking evaluation may be needed, so we need to define beta reduction. Remember that in Statix, beta head reduction is a relation with the signature:
\begin{lstlisting}
betaReduceHead : 
	(scope * Expr) * list((scope * Expr)) -> (scope * Expr)
\end{lstlisting}

In Haskell, a slightly different structure was used. We defined two functions, one returning a \verb|Maybe SExpr| and the other checking if the result \verb|Nothing|, in which case it returns the original expression (unreduced). This structure could also be implemented in Statix, but Haskell makes it a bit easier for us by providing the \verb|Maybe| type and functions on it.
\begin{lstlisting}
brh :: SExpr -> SExpr
brh e = fromMaybe e (brh_ e [])
brh_ :: SExpr -> [SExpr] -> Maybe SExpr
\end{lstlisting}

Now we can define beta head reduction as cases of \verb|brh_|, returning \verb|Nothing| if the expression cannot be reduced, in which case \verb|brh| will take the original expression. The cases correspond directly to the cases of figure \ref{fig:beta-head-reduce-rules}.

\section{Comparison}

The base implementations of the language are quite similar. One difference is the way the definitions are spread. In Statix, one could put each language construct in a separate file, keeping the definitions for that construct together. In the Haskell implementation this is not easily possible\footnote{We could still define a function that then calls the actual implementation in the separate files, but this is still inferior to the Statix implementation.}, since function definitions can not be split over a file. 

Another advantage that Statix has is that it has first-order inference built-in, which makes implementing a basic form of inference as described in chapter~\ref{ch:inference} way easier. However, if we want a more complex form of inference then an implementation in Haskell would be better, since it is a more expressive language and it has more libraries available. 

Finally, creating the language in Spoofax automatically gives us editor services such as code highlighting and semantic autocompletion, as dicusssed in chapter~\ref{chap:editor-services}. Implementing these in Haskell would require some work.
