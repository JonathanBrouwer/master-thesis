\chapter{Conclusion}
\label{ch:conclusion}

This thesis presents an implementation of a dependently typed language in Statix. Our aim was to answer the following research questions:

\paragraph{RQ1: Can the Calculus of Constructions be implemented in Statix?}
As we demonstrated in chapter~\ref{chap:baselang}, the Calculus of Constructions can be implemented concisely in Statix, by storing substitutions in the scope graph. We defined the beta-reduction, beta-equality and type checking rules and then converted them to Statix code. Beta reduction was defined using a Krivine machine. In chapter~\ref{chap:namecolls} we solved the variable capture problem which the naive implementation from chapter~\ref{chap:baselang} suffered from, by using scoped names.

\paragraph{RQ2: Is the implementation is easily extendable?}
We showed in chapter~\ref{chap:bools} that the implementation is easily extendable, by extending it with booleans, postulate, and type assertions. We define a four step process that can be used to extend the language, which we also use to answer the remaining research questions.

\paragraph{RQ3: Can we add inference to the implementation?}
In chapter~\ref{ch:inference} we discuss how to add inference to the implementation. If we want to keep the implementation clean and concise, we need to compromise on how powerful the inference algorithm is, by defining an approximated version of first-order inference. We implemented this algorithm in a concise way. In chapter~\ref{ch:comp-lambdapi} we explore how we could define the language in a language supporting equational unification, adding this to Statix would allows for a more powerful inference algorithm to be cleanly implemented.

\paragraph{RQ4: Can we add support for inductive data types to the implementation?}
chapter~\ref{ch:datatypes} adds support for inductive data types with parameters and indices. We give steps for type-checking data type declarations. Additionally, we give each data type an eliminator and show how to type-check and beta reduce eliminators. Finally, we show that positivity checking can be implemented concisely.


\paragraph{RQ5: Can we add support for universes to the implementation?}
In chapter~\ref{ch:universes} we showed that we can add universes to the language. This can be done easily and concisely.

\section{Future Work}
While the language as implemented currently is fully usable, there are still some open questions. 

\subsection*{Adding more features to the language}

In this thesis we created a language that already has a lot of features: Inductive datatypes, universes, and a basic inference algorithm. In chapter~\ref{chap:bools} we also specified the general steps that are needed to extend the language with new constructs. However, in comparison with languages such as Agda a lot of features are still missing, among which are:

\begin{itemize}
	\item Support for universe polymorphism~\cite[Universe Levels]{agda}, which allow functions and inductive data types to be polymorphic over a certain universe level. For example, this allows to define a \verb|List| data type that can be both a list of values and a list of types without defining it twice. Agda has quite a rather involved definition of equality between universe levels, which may be difficult to implement in Statix, but at least a simple version of universe polymorphism should be possible to implement in Statix.
	\item Another feature of Agda is dependent pattern matching, as described by Coquand~\cite{patternmatching}. Dependent pattern matching is complex because it requires interleaved type checking and desugaring, which is problematic since desugaring is usually done in Stratego, while type checking is done in Statix. The solution may be to do the desugaring in Statix instead, but at some point this is not ideal, since Statix is not meant for this. More research is needed on a good solution for this problem.
	\item Inference in our language is currently implemented by explicitly specifying a placeholder \verb|_| where inference is required. In Agda we can specify an implicit argument, which is automatically inferred unless explicitly provided. It can be tricky to determine when an implicit argument has to be inserted, since this may require some type information~\cite{implicit_arguments}. More research is needed on whether the solution from this paper can be implemented in Statix.
	\item Instance Arguments are a special kind of implicit arguments, that are solved by the instance resolution algorithm of Agda~\cite[Instance Arguments]{agda}. They can be used similarly to type class constraints, in order to restrict the data types over which a term is polymorphic, for example requiring the terms to have an implementation of equality. The instance resolution algorithm should be possible to implementing by traversing the scope graph (since it is linear), and checking whether each let binding is a possible solution similar to the Agda implementation. 
	\item Agda has holes, which allows the user of the language to build proofs interactively. One of these interactions is proof search, when triggered Agda will try to find a term that fits the expected type of a hole. The interactive part of this was discussed in chapter~\ref{ch:relatedwork}, this is not yet supported by Spoofax. After support for this is added, a proof search algorithm could be implemented in Stratego, such as the structurally recursive unification algorithm described by McBride~\cite{proofsearch}. The holes system that was introduced for semantic code completion~(chapter~\ref{chap:editor-services}) could be re-used for this purpose. Running the algorithm may require running type checking code, so here we encounter the interleaving problem again. More research is needed on whether proof search can be implemented in Spoofax.
	\item Agda allows for compile-time irrelevance annotations, which are a marker stating that the value of a certain variable is irrelevant, only that the type is non-empty matters~\cite[Irrelevance]{agda}. This can simplify some proofs, since all values of this type are considered equal. We see no reason why this couldn't be implemented in Statix.
	\item Agda allows function arguments to be marked as erased, meaning that the value will be erased at runtime. This is different than irrelevance annotations, as the value is still available at compile time. An even stronger version of this feature is implementing a quantitive type system, such as in Idris~2~\cite{idris2}, which combines linear and dependent types. Implementing this in Spoofax should be possible, but it may be quite a bit of work as it is a complex type system.
	\item Agda has mixfix operators, which allows to define custom operators such as \verb|a and b|~\cite[Mixfix Operators]{agda}. This requires parsing to happen in multiple stages, where the first stage finds the mixfix operator definitions and the second stage parses expressions using them~\cite{mixfix}. This results in quite a complex parser, implementing these multiple stages in Spoofax using SDF3 is not possible without fundamental changes to the way that Spoofax works. One solution would be using an \emph{adaptive parsing algorithm}, which allows modification of the grammar during parsing. More research is needed to explore the possibilities of solving this problem.
	\item Another thing that could be explored is creating a module system for a dependently typed language in Spoofax. At the time of writing, Agda has a module system which is quite powerful (including parameterized modules), but the implementation is quite naive by substituting the modules into the code when imported~\cite[Modules]{agda}. Spoofax's scope graphs allow to create a module system with ease, simply by adding extra \emph{import} edges between scopes. Future research is needed on if Agda's module system could be implemented using scope graphs.
	\item Currently the language in the paper does not allow for recursive functions. Agda allows for functions that are structurally recursive (as well as mutually structurally recursive), and uses termination checking to enforce this property. We see no reason why this couldn't be cleanly implemented in Statix, similar to the implementation of positivity checking in chapter~\ref{ch:datatypes}.
\end{itemize}

\subsection*{Implementing other complex type systems in Statix}

In this paper we have shown that a dependent type system can be implemented in Statix, one of the goals being to show the power of Statix. To further show this goal, more different kinds of type systems could be implemented in Statix:

\begin{itemize}
	\item There has not yet been an implementation of a sub-structural type system in Spoofax. A sub-structural type system is a type system that places restrictions on how often variables can be used. Now that languages with sub-structural type systems, such as Rust, are becoming popular, we should see if the goal of Statix to cover a broad range of type systems, still holds true for these. 
	\item Effect systems are a formal system that describe the side effects of a program. These allow programmers to specify what side effects a function has, which can then be verified by a type checker. Researching whether an effect system could be cleanly specified in Statix may lead to more insights on the limits of Statix.
	\item As mentioned above, implementing a quantitive type system, such as in Idris~2~\cite{idris2}, would be an interesting extension to this thesis. 
\end{itemize}